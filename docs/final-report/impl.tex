During our early research on Microservices, we found a number of online documents about Microservices, but most of them only explain the architecture style of Microservices which more accessible but remain too high-level to guide implementation. For example, searching “Microservices” on Google gives around 1,700,000 results, “Microservices example” only gives about 706,000 results. Furthermore, “Microservices data consistency” gives only 62,300 results. Our project aims to find out more about what technology can be used in implementing Microservices with data consistency guarantees, which is a relatively less discussed topic with fewer alternatives in terms of concrete implementations.

\hl{\textbf{Which systems we have chosen and why: }} Having decided to investigate Sagas, we looked into the existing implementations of Saga pattern for microservices. Among them, we picked \textbf{Eventuate} and \textbf{Axon}, which are the most popular and well maintained. We summerize the two frameworks briefly here:

\begin{itemize}
    \item \textbf{Axon framework} \cite{axon} is a Java based framework for building scalable and highly performant applications. The main notion is the event processing which includes the separated Command bus for updates and the Event bus for queries. According to an original Axon author’s response on StackOverflow \cite{comparison}, Axon has been around for about 8 years and is being used by many systems in production since then. Axon has extensive support for Spring. Yet Spring is not required for Axon, it made configuration easy with Spring annotations.
    \item \textbf{Eventuate} \cite{eventuate} is a platform that provides an event-driven programming model that focus on solving distributed data management in microservices architectures. The framework stores events in the MySQL database and it distributes them through the Apache Kafka platform. Eventuate is a framework that has integration with Gradle and Maven project. Therefore, we have implemented our prototype using Spring and Maven.
\end{itemize}

There are already comparisons online of the two frameworks \cite{comparison, stefanko}. All of we could gathered, however, were written by implementers of transaction frameworks. For example, \cite{stefanko} was written by the a member of the Narayana team within Red Hat, which also has a similar implementation. Therefore, in this project, we would like to evaluate the end-user experience of using Saga frameworks to implement data consistency in microservice systems.

\hl{\textbf{How we obtained the systems:}} Since both Eventuate and Axon frameworks themselves are freely available on the internet (it seems that some of other related services are not, we obtained them through their official websites \cite{axon, eventuate}.
