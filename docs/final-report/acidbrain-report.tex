%-----------------------------------------------------------------------------
%
%               Template for sigplanconf LaTeX Class
%
% Name:         sigplanconf-template.tex
%
% Purpose:      A template for sigplanconf.cls, which is a LaTeX 2e class
%               file for SIGPLAN conference proceedings.
%
% Guide:        Refer to "Author's Guide to the ACM SIGPLAN Class,"
%               sigplanconf-guide.pdf
%
% Author:       Paul C. Anagnostopoulos
%               Windfall Software
%               978 371-2316
%               paul@windfall.com
%
% Created:      15 February 2005
%
%-----------------------------------------------------------------------------

% \documentclass[11pt, nocopyrightspace, numbers]{assets/sigplanconf}
\documentclass[journal,11pt,onecolumn]{assets/IEEEtran}
% \documentclass[journal,11pt,onecolumn,draftclsnofoot]{assets/IEEEtran}

% The following \documentclass options may be useful:

% preprint      Remove this option only once the paper is in final form.
% 10pt          To set in 10-point type instead of 9-point.
% 11pt          To set in 11-point type instead of 9-point.
% numbers       To obtain numeric citation style instead of author/year.

\newcommand{\nocaptionrule}{ }
% \usepackage{caption} % http://ctan.org/pkg/caption


\usepackage{nimopaper}

\begin{document}

\title{\textsc{ACIDBrain}: Maintaining data consistency in microservices}

\author{
    \IEEEauthorblockN{
        Nana Pang\IEEEauthorrefmark{1},
        Wode ``Nimo'' Ni \IEEEauthorrefmark{2}, and
        Xin Chen \IEEEauthorrefmark{3}
    }

    \IEEEauthorblockA{
        Columbia University\\
        Email:
        \IEEEauthorrefmark{1}np2630@columbia.edu,
        \IEEEauthorrefmark{2}wn2155@columbia.edu,
        \IEEEauthorrefmark{3}xc2409@columbia.edu
    }
}

\maketitle


\begin{abstract}
    As software systems become more distributed and larger in scale nowadays, there is a growing need for novel architectures that are modularized, flexible, and scalable. Microservice architecture emerged as a popular software architecture where developers divide a large application into small, self-contained components (“services”). Although the separation significantly increases flexibility and reduces difficulty in development, maintaining data consistency across these isolated databases can still be be challenging. Our project targets maintaining data consistency among distributed microservices. To maintain data consistency, an event driven model called \textbf{Saga}\cite{garcia1987sagas} is often used to keep track of the data sources of each transaction. In this project, we investigate two predominant event sourcing frameworks: \textit{Eventuate} and \textit{Axon}, from an \textbf{end-user’s perspective}. We experiment with both of them by building data consistency guarantees on a dummy microservice system, and analyze the effectiveness of the two frameworks both quantitatively and qualitatively.
\end{abstract}


%------------------------------------------------------------------------------
% Content

\section{Introduction}
Large software systems have been growing rapidly and becoming more distributed. As a result, the traditional monolithic software architecture, where modules of different functionalities are closely coupled and developed by the same group of developers, is now challenged by a new architectural pattern: \textbf{Microservices}. A software system that employs microservice architecture comprises of a suite of clearly defined small services, each running in its own process \cite{lewis2014microservices}. As a result, each service can be developed by completely separated teams and using distinct technology stack, thereby decoupling the modules and parallelizing development.

In a monolithic system, all modules typically share the same backend database and enjoy the ACID (Atomicity, Consistency, Isolation, Durability) properties provided by the database \cite{gray1981transaction}. Under microservice architecture, however, each module would use a separate database to ensure good isolation among service modules, and each database might be using an entirely different technology. Therefore, large transactions that involves multiple modules, which used to run in the same database, now span across multiple databases with drastically different implementations. In this case, maintaining data consistency becomes a more complex task. \textbf{Two-phase-commit (2PC)} protocol \cite{bernstein1987concurrency} is a popular solution, where a coordinator process ensures all databases have the requested resource available before commiting the transaction. While 2PC ensures the correctness of distributed transactions, it requires all resources to be exclusively locked until the transaction finishes.

Unfortunately, in real world applications, many distributed transactions are long-lived, meaning they take much longer (in terms of hours and/or network gaps) to finish. For example, an e-commerce application may have ordering, billing, and shipping modules, and a complete transaction of purchasing an item would involve all components and would not complete until the item is shipped. In the case of these long-lived transactions, 2PC does not scale well because it locks all databases involved in the transaction and there can be many other transactions occurring in the system simultaneously.

Among the existing solutions to maintain data consistency in distributed databases of microservice systems, we chose to look at the \textbf{Saga pattern} \cite{garcia1987sagas}. There are a few arguments for Sagas over traditional consensus protocols such as 2PC. Firstly, the Saga pattern does not require synchronization of all databases in a transaction, making it more suitable for long-lived transactions. Also, Saga tends to be easier to implement than 2PC, whose logic is relatively complex.


\section{Background}
A saga is a sequence of local transactions. Each local transaction updates the database and publishes an event to trigger the next local transaction in the saga \cite{richardson2014saga}. Using this pattern, a large distributed transaction is broken into multiple local transactions that update their local databases and publish events globally to notify others, which are then coordinated by a saga. To maintain data consistency, each local transaction theoretically must be accompanied by compensatory transaction. As a result, the saga can rollback the large transaction by executing these compensations in a reverse order.

In general, there are two types of sagas: \textit{choreography-based} and \textit{orchestration-based}. An orchestration-based saga is a standalone object that coordinates multiple service in a centralized manner, whereas choreography-based saga is implemented by each local transaction publishing domain events that trigger local transactions in other services \cite{richardson2014saga}.

The use of microservices and sagas also motivates the overall architectural pattern to deviate from the traditional layered architecture. \textbf{Event sourcing} and \textbf{Command Query Responsibility Segregation (CQRS)} are often used to fully implement sagas. Since sagas coordinate local transactions by publishing events after updating local datastores, it is vital for the publication and update to be atomic - otherwise the system ends up in an inconsistent state. Event sourcing is used to enforce atomicity of events publishing and datastore updates. Event sourcing persists the state of entity as a sequence of state changing events. Whenever an entity changes, a new event is appended. The final state of each entity must be consistent since adding a single event is atomic. Therefore, we don’t need compensation rollback to maintain data consistency but persist events in an event store, adding or retrieving event when an entity is changed.

Unfortunately, with multiple local databases involved in one event store, querying these databases becomes much more difficult to implement than in a monolithic system. Since the current state of all entities are stored as a sequence of changing events rather than a static record, it is difficult to obtain the most up-to-date state of a certain entity. To mitigate the complexity, the CQRS architecture is often used in implementations of sagas. CQRS separates queries from commands/modifications to the datastore


\section{Conclusion}
In this project, we experimented with and evaluated two existing frameworks, Axon and Eventuate, that implements data consistency for distributed transactions using Saga pattern. The two frameworks are similar in principle. Both of them implement Saga pattern and CQRS view. From a end-user’s point of view, however, there are identifiable tradeoffs in choosing either of the frameworks. In short, Eventuate, being the newer design of the two, is more user-friendly and provides better decoupling of components in the system. However, comparing to Axon, it is more unstable and performs worse due to its shorter development time. Axon, on the other hand, has a less-friendly API with tighter coupling of components such as handlers for commands and events. Moreover, between the two frameworks, Axon is relatively more light-weight because it only provides a skeleton structure, whereas Eventuate provides functional implementation for most parts of the system and therefore contains more scaffolding code.


%------------------------------------------------------------------------------
% bibliography

\FloatBarrier{}
% prevent figures breaking the reference section

\bibliographystyle{unsrt}
\bibliography{bibliography}

\end{document}
